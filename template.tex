\documentclass[12pt,openright,twoside,a4paper,brazil,english,emptypage,openany]{abntex2}


% pacote que gera textos aleatorios
\usepackage{lipsum} % apagar

% -- ESTRUTURA BASICA -- %
\usepackage[alf,abnt-etal-cite=2]{abntex2cite}
\usepackage[a4paper, margin=3cm]{geometry}
\usepackage{appendix}
\usepackage{caption}
%\addbibresource{references.bib}

% -- ESTETICA -- %
\usepackage{fancyhdr}% 
\pagestyle{fancy}
\usepackage[flushleft]{threeparttable}
%\usepackage[sc]{mathpazo} % fonte diferente pra símbolos matemáticos

%% -- CREATE TABLE OF CONTENTS -- %%

% Adjust spacing for chapters in the table of contents
\cftsetindents{chapter}{3em}{2em}

% Adjust spacing for sections in the table of contents
\cftsetindents{section}{5em}{2em}

% Adjust spacing for subsections in the table of contents
\cftsetindents{subsection}{5em}{3em}


% -- MARGENS -- %

% For better justification
\usepackage{ragged2e} 

%make itemize respect margins
\usepackage{enumitem} 

% Adjust itemize margins
\setlist[itemize]{leftmargin=1.5em, labelsep=0.5em, itemsep=0.5em}



% -- FONTES e TITULOS -- %
\usepackage{titlesec} %titulos: deixa letras gordas e titulos em negritos


% Formatting chapter titles
\titleformat{\chapter}[display]
  {\sffamily\bfseries\large}{\thechapter}{-12pt}{\large\raggedright}
\titlespacing*{\chapter}{0pt}{-50pt}{40pt}

% Formatting section titles
\titleformat{\section}[display]
  {\sffamily\bfseries\normalsize}{\thesection}{-10pt}{}
\titlespacing*{\section}{0pt}{10pt}{0pt}

% Formatting subsection titles
\titleformat{\subsection}[display]
  {\sffamily\bfseries\normalsize}{\thesubsection}{-10pt}{}\titlespacing*{\subsection}{0pt}{10pt}{0pt}

\usepackage{graphicx} %inserir imagens

\hypersetup{
pdftitle={\@title},
pdfauthor={\@author},
pdfsubject={\imprimirpreambulo},
pdfkeywords={PALAVRAS}{CHAVES}{EM}{PORTUGUES},
pdfcreator={LaTeX with abnTeX2},
colorlinks=true,
linkcolor=black,
citecolor=blue,
urlcolor=black
}

% Table of contents
\makeatletter
\renewcommand{\tableofcontents}{%
  \chapter*{\contentsname}
  \@starttoc{toc}
}


\begin{document}
\pretextual
\selectlanguage{english}

% Set \parskip to put 10pt between paragraphs
\setlength{\parskip}{10pt}


% Folha de rosto
\pagenumbering{arabic}

%\pagestyle{headings} %cabeçalho
\pagestyle{fancy}% Change page style to fancy

%\renewcommand{\chaptermark}[1]{\markboth{#1}{}}
%\renewcommand{\sectionmark}[1]{\markright{#1}{}}

\fancyhf{}% Clear header/footer
\fancyhead[LE,LO]{\textit{\sffamily{\nouppercase{\newlinetospace{{\leftmark}}}}}}
\fancyhead[RE,RO]{\thepage}

\flushright
\thispagestyle{empty}%tirando numero da primeira pagina
\includegraphics[width=10cm]{logo_puc.png}

\vspace{20pt}

\large{Nome Aluno}

\vspace{100pt}

\Large{\textbf{\sffamily Título da Mono}}

\vspace{90pt}

\normalsize
Monografia de Final de Curso

\vspace{2pt}

Orientador: Nome Orientador

\vspace{40pt}

\centering

Declaro que o presente trabalho é de minha autoria e que não recorri, para realizá-lo, a nenhuma forma de ajuda externa, exceto quando autorizado pelo professor tutor.

\vspace{60pt}

Rio de Janeiro, Junho de 2023

\flushleft
\pagebreak




\section*{\large Agradecimentos}
\justifying

\lipsum[1]

\newpage

\section*{\large Abstract}
\justifying

\lipsum[1]

\vspace{4\onelineskip}
\noindent
\section*{\large Keywords}
\hspace{1em} oi; tudo bem.


%  SUMARIO
\sffamily
\tableofcontents



% INICIO DOS CAPITULOS
\rmfamily % voltando a usar fonte default

\chapter{Primeiro Capitulo}
\justifying

\lipsum[1]

\cite{elliott2013complete} suggests that... and Breiman(\citeyear{elliott2013complete}) says that ...
\citeonline{elliott2013complete}

\section{Subcapitulo}
\justifying

\lipsum[1]

\subsection{Subsubcapitulo}
\justifying

\lipsum[1]

\lipsum[1]

\lipsum[1]

\lipsum[1]

\chapter{Segundo Capitulo}
\justifying

\lipsum[1]

\bibliography{references}


% APENDICE %

\appendix

\chapter{Nome do primeiro apendice}
\justifying

\end{document}
